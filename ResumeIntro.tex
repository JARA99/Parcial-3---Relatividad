\begin{abstract}
    NO SE NOS OLVIDE HACER EL RESUMEN
\end{abstract}

\maketitle

\section{Introducción}

Un aparente problema, surge cuando tenemos dos partículas entrelazadas, las alejamos, y medimos una de ellas, colapsando el estado de ambas. 
Aparentemente, la información ha viajado de forma instantánea, violando las leyes de la relatividad especial. 

Hacer un análisis matemático al respecto es importante, y así dar una justificación al por que se busca la aprobación de una teoría: la mecánica cuántica no es local.

Esta no es más una teoría, ya que ha sido demostrada en disversas situaciones. Pero demostrar que los eventos no tienen una relación causal, es un buen ejercicio que nos da la dirección a la inminente conclusión que la mecánica cuántica es una teoría no local, y por tanto, no contradice a la relatividiad especial.

