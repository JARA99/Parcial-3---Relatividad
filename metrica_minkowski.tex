
	\section{Métrica Minkowski}
	
	El espacio-tiempo de Minkowski es un conjunto de cuatro dimensiones, con elementos etiquetados por tres dimensiones espaciales y una temporal. Un punto individual en el espacio tiempo es llamado un evento. La trayectoria de una partícula es una curva a través del espacio-tiempo. 
	
	
	
	EL intervalo entre dos eventos en el espacio tiempo está descrito:
	$$(\Delta s)^2=-(c\Delta t)^2+(\Delta x)^{2}+(\Delta y)^{2}+(\Delta z)^{2}$$
	
	
	
	
	donde $c$ es la velocidad de la luz en el vacío. Lo inportante en está definición de intervalo del espacio-tiempo entre dos eventos es que es \textit{invariante bajo trnasformaciones de coordenadas inerciales}. No existe una noción absoluta de "eventos simultáneos"; es decir si dos cosas ocurren al mismo tiempo depende de las coordenadas utilizadas.

	El espacio-tiempo tiene un tensor métrico asociado que puede escribirse en forma matricial como
$${\displaystyle \left(\eta _{\alpha \beta }\right):={\begin{pmatrix}-1&0&0&0\\0&1&0&0\\0&0&1&0\\0&0&0&1\end{pmatrix}}} $$
	
%{\overset {\underset {\mathrm {def} }{}}{=}}
Ahora el intervalo del espacio tiempo entre dos eventos escrito en forma tensorial:


	$$(\Delta s)^{2}= -\eta_{\mu \nu }\Delta x^{\mu}\Delta x^{\nu}$$
	

Una herramienta muy útil para comprender el espacio-tiempo es la estructura del cono de luz que están dividos en futuro y pasado. Todos los punto dentro del cono de luz futuro y pasado de  un evento <<O>> en el espacio tiempo, son llamados puntos \textit{timelike} con $(\Delta s)^{2} > 0 $. Y  los puntos de los conos son \textit{light-like} o nulos $ (\Delta s)^{2}=0 $ .Si se supone que todos los procesos causales se propagan a la velocidad de la luz o a una velocidad menor, se concluye que estos son todos los eventos que se pueden afectar causalmente a partir de O. 
%El cono de luz futuro en O contiene todos los eventos en el espacio-tiempo a los que se puede llegar desde O mediante curvas de luz o de tiempo dirigidas por el futuro. 

Los puntos que  están fuera del cono de luz del evento O están separado en forma \textit{space-like} con $ (\Delta s)^{2}<0 $.  Si se asume que ningún proceso causal se propaga más rápido que la luz, estos eventos están causalmente desconectados de O. 



%\input{EPR-Bell.tex}


	

	
